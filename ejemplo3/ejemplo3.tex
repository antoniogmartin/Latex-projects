\documentclass[a4paper,11pt]{article}

\usepackage[utf8]{inputenc}

\usepackage[spanish]{babel}
\parindent=50cm
%\parskip=0.5cm
%\usepackage{anysize} para cambiar márgenes de la página
%\usepackage{geometry} permite modificar la disposición de todos los elementos de la página.



\begin{document}%inicio entorno
  \title{Articulo de prueba}

  \author{Antonio Guzmán}

  \date{25/04/16}
\maketitle% Sin esto NO SE PUEDE GENERAR EL TITULO CON LOS DATOS DE ARRIBA
\tableofcontents %indice
%%%%%%%SECCIONES Y PARTES%%%%%%%%%%%%%%%%
\section{Prólogo}
Esto es un texto normal en el propio body, seguido de algo de relleno sin contenido intelectual real (salvo el meramente textual)
 para hacer un poco de bulto y que no se vea tan soso ni tan cortito.
\part{Comando part y section} %apartado principal

Texto de relleno bla bla bla bla bla bla bla bla bla bla bla bla bla bla bla bla bla bla bla bla bla bla bla bla

\section{Una sección}%seccion

Texto de relleno bla bla bla bla bla bla bla bla bla bla bla bla bla bla bla bla bla bla bla bla bla bla bla bla

\subsection{Una subsección dentro de la sección}%subseccion

Texto de relleno bla bla bla bla bla bla bla bla bla bla bla bla bla bla bla bla bla bla bla bla bla bla bla bla

\section{Otra sección}

Texto de relleno bla bla bla bla bla bla bla bla bla bla bla bla bla bla bla bla bla bla bla bla bla bla bla bla

\section{Y otra sección}

Texto de relleno bla bla bla bla bla bla bla bla bla bla bla bla bla bla bla bla bla bla bla bla bla bla bla bla

\begin{quote}%cita

Y esto es un texto en un entorno de quote, seguido de algo de relleno sin contenido intelectual real (salvo el meramente textual)
 para hacer un poco de bulto y que no se vea tan soso ni tan cortito.

\end{quote}

\begin{flushright}%alineado derecha (flushleft para la izq)

Y esto otro es un texto en un entorno flushleft, seguido de algo de relleno sin contenido intelectual real (salvo el meramente textual)
 para hacer un poco de bulto y que no se vea tan soso ni tan cortito.

\end{flushright}
%%%%MATEMATICAS%%%%%%%%%%%%%%%%%%%%%%%%%%%%%%%%%%%%%%%%%%%%
\part{formulas matemáticas}
La ecuacion mas famosa de la historia de la fisica probaablemente sea la de $E=m*c^2$, de donde se deduce que $c=\sqrt{E/m}$.
\section{Display math}

\begin{displaymath}
\sum_{0\le i\le m\\0<j<n}P(i, j)% forma de crear un sumatorio
\end{displaymath}

Hay muchos simbolos y toda un compleja sintaxis para escribir matematicas en LaTeX

\begin{equation}%ecuacion
\label{miecuacion}% Usado para hacer referencias en el documento (1)
f(x)=\sqrt{g'(x)dx}+Z
\end{equation}

La ecuacion que se puede ver en\ref{miecuacion} es completamente inventada y no tiene sentido fisico%/ref se asocia con la referencia 1

\part{Acentos y caracteres especiales}
\'a \'e
\section{la e\~ne}
se escribe \begin{quote}'\~n' ¿esto que es:AHH son los puntos suspensivos\ldots?
\end{quote}

\section{Caracteres especiales}
\label{sec:caracteres especiales}
Los carácteres especiales deben ir precedidos de:  no debemos poner \textbackslash\ \textbackslash\ % asi se declara \ con \textbackslash\
porque eso se usa para dar un enter
\{, \},\$,\% %asi se representan caracteres especiales
%%%%%%%%%%%%%%%TIPOS DE LETRA%%%%%%%%%%%%%%%%%%%%%%%%%%%%%%%%%%%%%%%
\part{Tipos de letra}
Cuando introducimos en el texto uno de estos comandos, modificará todo el texto subsiguiente, hasta que llegue a otro comando que vuelva a cambiarlo\\
Son los siguientes tipos:\\
\section{Para textos cortos}


{\rm Roman} (redonda seriff, la tipografía normal).//
\bf Boldface (negrita)

\em Italic (cursiva).

\bf Boldface (negrita).

\sl Slanted (inclinada).

\sf Sans Serif (sin seriff, de 'palo seco').

\sc Small Caps (todas mayúsculas, solo cambia el tamaño).

\tt Typewriter (De paso 'fijo'  o monotype).%%%%%%%%%%%%%%%%%% Y ESTA
\rm Roman% esta es la ultima tipografia que se va a aplicar al resto del documento
 \section{Para textos largos}

 \begin{rm}
 Roman (redonda seriff, la tipografía normal). Usando un entorno, que es más cómodo en estas circunstancias.
 \end{rm} \\

\begin{em}
Italic (cursiva). Usando un entorno, que es más cómodo en estas circunstancias.
\end{em}

\section{Tamaños de letra}
\begin{rm}
%Tamaños de letra disponible  \tiny, \scriptsize, \footnotesize, \small, \normalsize, \large, \Large, \LARGE, \huge y \Huge.%
\scriptsize\ texto peque
\itshape\ texto pequeño en cursiva.
\end{rm}

\begin{Large}

Texto en largue y, por tanto, grandote

\end{Large}

\part{Unidades de medida}

mm (Milímetro) 1/1000 de metro en el sistema métrico decimal.
cm (Centímetro) 1/100 de metro en el sistema métrico decimal.

in (Pulgada) Equivalente a 2.54 cm. Muy popular en el ámbito anglosajón.

pt (Punto) Equivale a 0.351 mm y es una de las más clásicas unidades de medida tipográficas.
pc (Pica) 12 pt

bp (Big point) a 0.353 mm o 1/72 pulgadas (se le conoce como Punto Postscript)
dd (Didot Point) Antigua medida tipográfica francesa de 0.376 mm
cc (Cicero) 12 dd

sp (Scaled point) 1/65536 de pt ¡No es una errata!

Además de todas las anteriores, que son absolutas y su valor no cambia nunca, también hay un juego de unidades relativas, que dependen del tamaño de la tipografía que se esté usando:

ex (Equis) Altura de la letra “x” de la letra que se esté usando en un momento dado.
em (Eme) Anchura de la letra “M” de la letra que se esté usando en un momento dado.
mu (mu) 1/18 de em.

\part{Listas en Latex}
\section{listas con enumeración}
\begin{enumerate}
 \item Primer elemento de la lista
 \begin{enumerate}
   \item Subelemento de la primera lista
 \end{enumerate}
 \item Segundo elemento de la lista
 \item Tercer elemento de la lista
\end{enumerate}
\section{listas sin enumeración}
\begin{itemize}
 \item Primer elemento de la lista
 \begin{itemize}
   \item Subelemento de la primera lista
 \end{itemize}
 \item Segundo elemento de la lista
 \item Tercer elemento de la lista
\end{itemize}% ademas de ambos tipos de listas se pueden combinar enumerados con no enumerados
\part{Tablas}
\section{Tablas no flotantes}
\begin{tabular}{|c|c|c|}
\hline
A & B & C\\
\hline\hline
foo & bar & baz\\
\hline
zab & rab & oof\\
\hline
\end{tabular}
\\
\begin{tabular}{|c|c|}
\hline% linea de tabulación
Primera celda & Segunda celda \\
\hline
  Abajo a la izquierda & Abajo a la derecha \\
\hline
\end{tabular}
\section{Tablas flotantes}

\begin{table}%% esto indica que es tabla flotante
\begin{tabular}{|c||c|}
\hline
Primera celda & Segunda celda \\
\hline
Abajo a la izquierda & Abajo a la derecha \\
\hline
\end{tabular}
\end{table}
\part{Cajas}

\section{Sin marco}
\mbox{este texto NO está enmarcado}
\makebox[9cm][r]{Caja de 9 centímetros con texo a la derecha}%mas control

\section{Con marco}
\fbox{este texto Sí está enmarcado}
\framebox[9cm][s]{Caja de 9 centímetros con texo justificado}%mas control
\\
\framebox[0.5\linewidth][c]{Caja de 1/2 del ancho total}%linewidth ancho total de la linea

\section{Cajas con párrafos}

\begin{minipage}[b]{0.5\linewidth}% texto posicionado en el fondo de la minipage [b]
Una caja con minipage
\end{minipage}
 \part{Establecer márgenes}
 Para establecer márgenes necesitamos el paquete anysize y la orden \textbackslash\ marginsize (descomentar para usarlo)
% \marginsize{2cm}{2cm}{2cm}{2cm}%\marginsize{izquierda}{derecha}{arriba}{abajo}
\part{Identado y espacio entre párrafos}
Parident: para identado
Parskip:espacio entre párrafos
\section{FIN:continua en ejemplo 5}


\end{document}%fin entorno
