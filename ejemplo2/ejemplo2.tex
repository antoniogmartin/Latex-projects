\documentclass[a4paper,11pt]{article}

\usepackage[utf8]{inputenc}

\usepackage[spanish]{babel}





\begin{document}%inicio entorno
  \title{Articulo de prueba}

  \author{Antonio Guzmán}

  \date{25/04/16}
\maketitle% Sin esto NO SE PUEDE GENERAR EL TITULO CON LOS DATOS DE ARRIBA
\tableofcontents %indice
\section{Prólogo}
Esto es un texto normal en el propio body, seguido de algo de relleno sin contenido intelectual real (salvo el meramente textual)
 para hacer un poco de bulto y que no se vea tan soso ni tan cortito.
\part{Apartado Principal} %apartado principal

Texto de relleno bla bla bla bla bla bla bla bla bla bla bla bla bla bla bla bla bla bla bla bla bla bla bla bla

\section{Una sección}%seccion

Texto de relleno bla bla bla bla bla bla bla bla bla bla bla bla bla bla bla bla bla bla bla bla bla bla bla bla

\subsection{Una subsección dentro de la sección}%subseccion

Texto de relleno bla bla bla bla bla bla bla bla bla bla bla bla bla bla bla bla bla bla bla bla bla bla bla bla

\section{Otra sección}

Texto de relleno bla bla bla bla bla bla bla bla bla bla bla bla bla bla bla bla bla bla bla bla bla bla bla bla

\section{Y otra sección}

Texto de relleno bla bla bla bla bla bla bla bla bla bla bla bla bla bla bla bla bla bla bla bla bla bla bla bla

\begin{quote}%cita

Y esto es un texto en un entorno de quote, seguido de algo de relleno sin contenido intelectual real (salvo el meramente textual)
 para hacer un poco de bulto y que no se vea tan soso ni tan cortito.

\end{quote}

\begin{flushright}%alineado derecha (flushleft para la izq)

Y esto otro es un texto en un entorno flushleft, seguido de algo de relleno sin contenido intelectual real (salvo el meramente textual)
 para hacer un poco de bulto y que no se vea tan soso ni tan cortito.

\end{flushright}

\part{formulas matemáticas}
La ecuacion mas famosa de la historia de la fisica probaablemente sea la de $E=m*c^2$, de donde se deduce que $c=\sqrt{E/m}$.
\section{Display math}

\begin{displaymath}
\sum_{0\le i\le m\\0<j<n}P(i, j)% forma de crear un sumatorio
\end{displaymath}

Hay muchos simbolos y toda un compleja sintaxis para escribir matematicas en LaTeX

\begin{equation}%ecuacion
\label{miecuacion}% Usado para hacer referencias en el documento (1)
f(x)=\sqrt{g'(x)dx}+Z
\end{equation}

La ecuacion que se puede ver en\ref{miecuacion} es completamente inventada y no tiene sentido fisico%/ref se asocia con la referencia 1



\end{document}%fin entorno
